%%%%%%%%%%%%%%%%%%%%%%%%%%%%%%%%%%%%%%%%%
% Uppsala University Assignment Title Page 
% LaTeX Template
% Version 1.0 (27/12/12)
%
% This template has been downloaded from:
% http://www.LaTeXTemplates.com
%
% Original author:
% WikiBooks (http://en.wikibooks.org/wiki/LaTeX/Title_Creation)
% Modified by Elsa Slattegard to fit Uppsala university
% License:
% CC BY-NC-SA 3.0 (http://creativecommons.org/licenses/by-nc-sa/3.0/)

%\title{Title page with logo}
%----------------------------------------------------------------------------------------
%	PACKAGES AND OTHER DOCUMENT CONFIGURATIONS
%----------------------------------------------------------------------------------------

\documentclass[12pt]{article}

\usepackage{tikz,lipsum,lmodern}
\usepackage[most]{tcolorbox}

\usepackage[utf8]{inputenc}
\usepackage[greek,english]{babel}
\usepackage{alphabeta}
\usepackage{amsmath}
\usepackage{gfsartemisia}
\usepackage{graphicx}
\usepackage{subfig}
\usepackage{float}
\usepackage[colorinlistoftodos]{todonotes}
\usepackage{tabularx}
\usepackage[myheadings]{fullpage}
\usepackage{enumitem}
\PassOptionsToPackage{hyphens}{url}\usepackage{hyperref}
\usepackage{tikz}
\usepackage[nottoc]{tocbibind} %Adds "References" to the table of contents
\usepackage{xcolor} % to access the named colour LightGray
\definecolor{LightGray}{gray}{0.9}


\usepackage{tabto}
\usepackage{minted}


\hyphenpenalty=10000


\usepackage{geometry}
\geometry{
	a4paper,
	total={170mm,257mm},
	left=20mm,
	top=20mm,
}


\addto\captionsenglish{% Replace "english" with the language you use
	\renewcommand{\contentsname}%
	{Περιεχόμενα}%
}

\addto\captionsenglish{
	\renewcommand{\partname}{}
}
\renewcommand{\thepart}{}

\makeatletter
\renewcommand{\fnum@figure}{Εικόνα \thefigure}
\makeatother


\renewcommand{\H}{\textlozenge}

%=====fonts==========%
%\usepackage{libertine}


%====================%





%=============header + footer ======================%
%

\usepackage{fancyhdr}

\pagestyle{fancy}
\fancyhf{}
\rhead{Σύγχρονα θέματα Τεχνολογίας Λογισμικού \includegraphics[width=0.7cm]{UNIPI_(logo).png}}
\lhead{Πανεπιστήμιο Πειραιώς \\ Τμήμα Πληροφορικής}
\cfoot{Σελίδα \thepage}

% 

%\setlength\headheight{47pt}
%=====================================================%



%\setcounter{secnumdepth}{0} % sections are level 1

\begin{document}
	
	\begin{titlepage}
		
		\newcommand{\HRule}{\rule{\linewidth}{0.5mm}} % Defines a new command for the horizontal lines, change thickness here
		
		\center % Center everything on the page
		
		%----------------------------------------------------------------------------------------
		%	HEADING SECTIONS
		%----------------------------------------------------------------------------------------
		
		\textsc{\LARGE Πανεπιστήμιο Πειραιώς}\\[1.5cm] % Name of your university/college
		\includegraphics[scale=0.6]{UNIPI_(logo).png}\\[1cm] % Include a department/university logo - this will require the graphicx package
		\textsc{\Large Τμήμα Πληροφορικής}\\[0.5cm] % Major heading such as course name
		\textsc{\large Σύγχρονα θέματα Τεχνολογίας Λογισμικού - Λογισμικό για κινητές συσκευές}\\[0.5cm] % Minor heading such as course title
		
		%----------------------------------------------------------------------------------------
		%	TITLE SECTION
		%----------------------------------------------------------------------------------------
		
		\HRule \\[0.4cm]
		{ \Large \bfseries Αριστοτέλης Ματακιάς - Α.Μ: Π19100}\\[0.4cm] % Title of your document
		{ \Large \bfseries Βασίλη Γκιάτα - Α.Μ. Π19036}\\[0.4cm] % Title of your document
		\HRule \\[1.5cm]
		
		%----------------------------------------------------------------------------------------
		%	AUTHOR SECTION
		%----------------------------------------------------------------------------------------
		%
		
		\textsc{\Large Εργασία στο μοντέλο Model View Controller \\[0.4cm] Τεχνικό Εγχειρίδιο} % Minor heading such as course title
		
		% If you don't want a supervisor, uncomment the two lines below and remove the section above
		%\Large \emph{Author:}\\
		%John \textsc{Smith}\\[3cm] % Your name
		
		%----------------------------------------------------------------------------------------
		%	DATE SECTION
		%----------------------------------------------------------------------------------------
		
		
		
		\vfill % Fill the rest of the page with whitespace
		
	\end{titlepage}
	
	
	
	%\selectlanguage{greek}
	\tableofcontents
	%\selectlanguage{english}
	
	
	
	% \section*{Εισαγωγή}
	% Στο παρόν έγγραφο βρίσκονται οι απαντήσεις στις ασκήσεις της Δεύτερης Εργασίας του μαθήματος. Πρώτη παρουσιάζεται η άσκηση που αντιστοιχεί στο επώνυμο του φοιτητή, η οποία είναι και η υποχρεωτική. Στη συνέχειά παρουσιάζονται λύσεις για τις υπόλοιπες ασκήσεις της εργασίας. Για τις ασκήσεις αυτές έχει δοθεί συνοπτική περιγραφή.
	
	\newpage
	
	
	
	\section{Ζητούμενα}
	
	%Σε αυτή την εργασία στόχος είναι η δημιουργία εφαρμογής για Android με τίτλο \textbf{MyPOIs} (\textbf{My Points Of Interest}). Συγκεκριμένα, κατά την ενεργοποίησή της και με το πάτημα κατάλληλων κουμπιών θα υπάρχουν οι εξής επιλογές:
	
%	\begin{itemize}
%		\item Εισαγωγή νέου POI
%		\item Αναζήτηση POI
%		\item Επεξεργασία POI
%		\item Διαγραφή POI
%		\item Προβολή όλων
%	\end{itemize}

	\section{Student Controller}
	
	public  Student StudentGetter()\\
	
	Η μέθοδος StudentGetter αναζητά και επιστρέφει τον μαθητή που είναι συνδεδεμένος.\\
	Αρχικά αποθηκεύει το userid που έχουμε αποθηκεύσει στο httpcontext.session και αναζητάει τον μαθητή με το συγκεκριμένο userid. Τέλος, αποθηκεύει τον συνδεμένο μαθητή και συμπεριλαμβάνει μαζί τα μοντέλα courses και CourseHasStudent για να έχουμε στην πρόσβαση μας τους βαθμούς και τα ονόματα των μαθήματων του.
	
	public IActionResult Grades()\\
	Η μέθοδος αποθηκεύει τον συνδεδεμένο μαθητή και επιστρέφει το View με τον μαθητή ως παράμετρο. \\
	public IActionResult Total()\\
	Η μέθοδος αποθηκεύει τον συνδεδεμένο μαθητή και επιστρέφει το View με τον μαθητή ως παράμετρο. \\
	
	public IActionResult Home()\\
	Η μέθοδος αποθηκεύει τον συνδεδεμένο μαθητή και επιστρέφει το View με τον μαθητή ως παράμετρο. \\
	
	public IActionResult Semesters(int? page)\\
	Η μέθοδος έχει ως παράμετρο την σελίδα/εξάμηνο που έχει πατήσει ο μαθητής να δει τα μαθήματα του. Αποθηκεύει τον συνδεδεμένο μαθητή και έπειτα αναζητάει τα μαθήματα που είναι στο ίδιο εξάμηνο με την μεταβλητή page. Τέλος επιστρέφει το View με παράμετρο το μοντέλο CourseHasStudents.
	
	

	\section{Πηγές}
	
	\begin{itemize}
		\item Βίντεο για την δημιουργία ενός RecyclerView.\\
		\url{https://youtu.be/Mc0XT58A1Z4}
		
		\item Βίντεο για την υλοποίηση του OnClickEvent σε items του RecyclerView.\\
		\url{https://youtu.be/7GPUpvcU1FE}
		
		\item Android Documentation για την δημιουργία μενού στην μπάρα της εφαρμογής.\\
		\url{https://developer.android.com/develop/ui/views/components/appbar/actions}
		
		\item Βίντεο για την δημιουργία κουμπιού με progress bar.\\
		\url{https://youtu.be/zv9R5EcRKHM}
		
		\item Android Documentation για την δημιουργία του File Provider.\\
		\url{https://developer.android.com/reference/androidx/core/content/FileProvider}
	\end{itemize}
	
	
	
\end{document}